\let\negmedspace\undefined
\let\negthickspace\undefined
\documentclass[journal]{IEEEtran}
\usepackage[a5paper, margin=8mm, onecolumn]{geometry}
%\usepackage{lmodern} % Ensure lmodern is loaded for pdflatex
\usepackage{tfrupee} % Include tfrupee package

\setlength{\headheight}{1cm} % Set the height of the header box
\setlength{\headsep}{0mm}     % Set the distance between the header box and the top of the text

\usepackage{gvv-book}
\usepackage{gvv}
\usepackage{cite}
\usepackage{amsmath,amssymb,amsfonts,amsthm}
\usepackage{algorithmic}
\usepackage{graphicx}
\usepackage{textcomp}
\usepackage{xcolor}
\usepackage{txfonts}
\usepackage{listings}
\usepackage{enumitem}
\usepackage{mathtools}
\usepackage{gensymb}
\usepackage{comment}
\usepackage[breaklinks=true]{hyperref}
\usepackage{tkz-euclide} 
\usepackage{listings}
% \usepackage{gvv}                                        
\def\inputGnumericTable{}                                 
\usepackage[latin1]{inputenc}                                
\usepackage{color}                                            
\usepackage{array}                                            
\usepackage{longtable}                                       
\usepackage{calc}                                             
\usepackage{multirow}                                         
\usepackage{hhline}                                           
\usepackage{ifthen}                                           
\usepackage{lscape}
\begin{document}

\bibliographystyle{IEEEtran}
\vspace{3cm}

\title{2023-January-25 Shift-2}
\author{AI24BTECH11005 - Prajwal Naik}
% \maketitle
% \newpage
% \bigskip
{\let\newpage\relax\maketitle}

\renewcommand{\thefigure}{\theenumi}
\renewcommand{\thetable}{\theenumi}
\setlength{\intextsep}{10pt} % Space between text and floats


\numberwithin{equation}{enumi}
\numberwithin{figure}{enumi}
\renewcommand{\thetable}{\theenumi}

\begin{enumerate}
\setcounter{enumi}{15}
 
%16
    \item Let N be the sum of the numbers appeared when two fair dice are rolled and let the probability that $N-2,\sqrt{3}N,N+2$ are in geometric progression be $\frac{k}{48}$. Then the value of k is:
	    \hfill{[Jan 2022]}
        \begin{multicols}{1}
            \begin{enumerate}
                \item 2
                \item 4
                \item 16
                \item 8
            \end{enumerate}
        \end{multicols}

%17
    \item The integral $16\int_{1}^{2} \frac{dx}{x^3\brak{x^2+2}^2}$ is equal  to 

		\begin{multicols}{4}
			\begin{enumerate}
				\item $\frac{11}{6}+\ln{4}$
				\item  $\frac{11}{12}+\ln{4}$
				\item  $\frac{11}{12}-\ln{4}$
				\item  $\frac{11}{6}-\ln{4}$
			\end{enumerate}
		\end{multicols}

%18
    \item Let T and C be  the transverse and conjugate axes of the hyperbola $16x^2-y^2+64x+4y+44=0.$ Then the area of the region above parabola $x^2=y+4$, below the transverse axis T and on the right of the conjugate axis C is :
        \begin{multicols}{4}
            \begin{enumerate}
               \item $4\sqrt{6}+\frac{44}{3}$
               \item $4\sqrt{6}+\frac{28}{3}$
               \item $4\sqrt{6}-\frac{44}{3}$
               \item $4\sqrt{6}-\frac{28}{3}$
            \end{enumerate}
        \end{multicols}

%19
    \item Let $\overrightarrow{a}=-\hat{i}-\hat{j}+\hat{k}, \overrightarrow{a}.\overrightarrow{b}=1$ and $\overrightarrow{a}\times \overrightarrow{b}=\hat{i}-\hat{j}$. Then $\overrightarrow{a}-6\overrightarrow{b}$ is equal to 
		\begin{multicols}{1}
			\begin{enumerate}
				\item $3\brak{\hat{i}-\hat{j}-\hat{k}}$
    \item $3\brak{\hat{i}+\hat{j}+\hat{k}}$
    \item $3\brak{\hat{i}-\hat{j}+\hat{k}}$
    \item $3\brak{\hat{i}+\hat{j}-\hat{k}}$
			\end{enumerate}
		\end{multicols}

%20
    \item The foot of perpendicular of the point $\brak{2,0,5}$ on the line $\frac{x+1}{2}=\frac{y-1}{5}=\frac{z+1}{-1}$ is $\brak{\alpha,\beta,\gamma}$. Then Which of the following is not correct?
		\begin{multicols}{1}
			\begin{enumerate}
				
				\item $\frac{\alpha \beta}{\gamma}=\frac{4}{15}$
    \item $\frac{\alpha }{\beta}=-8$
     \item $\frac{\beta }{\gamma}=-5$
      \item $\frac{\gamma }{\alpha}=\frac{5}{8}$
			\end{enumerate}
		\end{multicols}

%21
    \item For the two positive numbers a,b, if a,b and $\frac{1}{18}$ are in a geometric progession, while $\frac{1}{a},10,\frac{1}{b}$ are in a arithmetic progression, then, $16a+12b$ is equal to :
%22
    \item Points $P\brak{-3,2},Q\brak{9,10},R\brak{\alpha,4}$ lie on a circle C with $PR$ as its diameter . The tangents to C at the points Q and R intersect at the point S. If S lies on the line $2x-ky=1$, then K is equal to 
%23
    \item Let $a\in R$ and let $\alpha,\beta$ be the roots of the equation $x^2+60^{\frac{1}{4}}x+a=0$. If $\alpha^4 +\beta^4=-30$, then the product of all possible values of a is .
\\
%24
    \item Suppose Anil's mother wants to give 5 whole fruits to anil from a basket of 7 red apples, 5 white apples and 8 oranges. If in the selected 5 fruits , at least 2 orange, at least one red apple and at least one white apple must be given , then the number of ways , Anil's  mother can offer 5 fruits to Anil is   .\\
%25
    \item If m and n respectively are the numbers of positive and negative values of $\theta$ in the interval $[-\pi,\pi]$ that satisfy the equation $\cos 2\theta \cos{\frac{\theta}{2}}=\cos 3\theta \cos{\frac{9\theta}{2}}$ \\,, then the value of mn is equal to .

    \item The number of numbers, strictly between 5000 and 10000 can be formed using the digits 1,3,5,7,9 without repetition, is \\
    \item The remainder when $\brak{2023}^{2023}$ is divided by 35 is \\

    \item If the shortest distance between the line joining the points $\brak{1,2,3}$ and $\brak{2,3,4}$ and the line $\frac{x-1}{2}=\frac{y+1}{-1}=\frac{z-2}{0}$ is $\alpha$, then $28\alpha^2$ is equal to . \\
    \item 25percent of the population are smokers A smoker has 27 times more chances to devlop lung cancer then a non-smoker. A person is diagnosed with lung cancer and the probability that this person is a smoker is $\frac{k}{10}.$ Then the value of K is .\\
    \item A traingle is formed by $X-axis,Y-axis$ and the line $3x+4y=60$. Then the number of points $P\brak{a,b}$  which lie strictly inside  the triangle, where a is an integer and b is a multiple of a, is .

 \end{enumerate}

\end{document}
