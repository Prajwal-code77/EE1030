%iffalse
\let\negmedspace\undefined
\let\negthickspace\undefined
\documentclass[journal,12pt,twocolumn]{IEEEtran}
\usepackage{cite}
\usepackage{amsmath,amssymb,amsfonts,amsthm}
\usepackage{algorithmic}
\usepackage{graphicx}
\usepackage{textcomp}
\usepackage{xcolor}
\usepackage{txfonts}
\usepackage{listings}
\usepackage{enumitem}
\usepackage{mathtools}
\usepackage{gensymb}
\usepackage{comment}
\usepackage[breaklinks=true]{hyperref}
\usepackage{tkz-euclide} 
\usepackage{listings}
\usepackage{gvv}                                        
%\def\inputGnumericTable{}                                 
\usepackage[latin1]{inputenc}                                
\usepackage{color}                                            
\usepackage{array}                                            
\usepackage{longtable}                                       
\usepackage{calc}                                             
\usepackage{multirow}                                         
\usepackage{hhline}                                           
\usepackage{ifthen}                                           
\usepackage{lscape}
\usepackage{tabularx}
\usepackage{array}
\usepackage{float}


\newtheorem{theorem}{Theorem}[section]
\newtheorem{problem}{Problem}
\newtheorem{proposition}{Proposition}[section]
\newtheorem{lemma}{Lemma}[section]
\newtheorem{corollary}[theorem]{Corollary}
\newtheorem{example}{Example}[section]
\newtheorem{definition}[problem]{Definition}
\newcommand{\BEQA}{\begin{eqnarray}}
\newcommand{\EEQA}{\end{eqnarray}}
\newcommand{\define}{\stackrel{\triangle}{=}}
\theoremstyle{remark}
\newtheorem{rem}{Remark}

% Marks the beginning of the document
\begin{document}
\bibliographystyle{IEEEtran}
\vspace{3cm}

\title{18. Definite Integrals and Applications of Integrals}
\author{ai24btech11005,Bhukya Prajwal Naik}
\maketitle
\section{MCQs with One Correct Answer}

\begin{enumerate}
\setcounter{enumi}{13}   
\item
\begin{equation}
If f(x) = 
\begin{cases}
	e^{\cos x}\sin x, & \text(for \abs{x} \leq 2)\\
 	2, & \text(otherwise)\\
\end{cases}
\end{equation}
then $\int_{-2}^{3}f\brak{x} dx$=
\begin{enumerate}
    \item 0
    \item 1
    \item 2
    \item 3
    \hfill{{2000S}}
\end{enumerate}
\item The value of the integral $\int_{e^-1}^{e^2}\left|\frac{\log_e x}{x}\right| dx$ is:
\begin{enumerate}
	\item $\frac{3}{2}$
	\item $\frac{5}{2}$
    \item  3
    \item  5
    \hfill{{2000S}}
\end{enumerate}
\item The value of $\int_{-\pi}^{\pi}\frac{\cos^2 x}{1+a^ x } dx$,$a>0$
\begin{enumerate}
    \item$\pi$
    \item $a\pi$
    \item $\frac{\pi}{2}$
    \item $2\pi$
    \hfill{{2001S}}
\end{enumerate}
\item The area bounded by the curves $y=\abs{-1} and y=-\abs{x}+1$ is
\begin{enumerate}
 \item  1
 \item  2
 \item  $\frac{2}{\sqrt{2}}$
 \item  4
 \hfill{{2002S}}
\end{enumerate}
\item Let$f\brak{x}=\int_{1}^{x}\sqrt{2-{t}^2} dt $.Then the real roots of the equation ${x}^2-f\brak{'x}=0$ are 
\begin{enumerate}
  \item ${1}$
  \item ${1}{\sqrt2}$
  \item $\frac{1}{2}$
   \item 0 and 1
   \hfill{{2002S}}
   \end{enumerate}
   \item Let $T>0$ be a real number. Suppose f is continuous function such that for all ${x}\in{R}$,f$\brak{x+T}=f\brak{x}.$
   If I=$\int_{0}^{T}f\brak{x}dx$ then the value of $\int_{3}^{3+3T} f\brak{2x} dx $ is
\begin{enumerate}
	\item $\frac{3}{2}{I}$
   \item 2I
   \item 3I
   \item 6I
   \hfill{{2002S}}
   \end{enumerate}
   \item The integral $\int_{-1/2}^{1/2} \left( \lfloor x \rfloor + \ln \left( \frac{1+x}{1-x} \right) \right) \, dx$ equal to
\begin{enumerate}[label=(\alph*)]
	\item $\frac{-1}{2}$
 \item 0
 \item 1
 \item $2ln\brak{\frac{1}{2}}$
 \hfill{2002S}
 \end{enumerate}



   \item If $l\brak{m,n}=\int_{0}^{1} t^{m}\brak{1+t}^{n} dt$,then the expression for l\brak{m,n} in terms of l\brak{m+1,n-1} is
\begin{enumerate}
 \item $\frac{2^{n}}{m+1} -\frac{n}{m+1}l\brak{m+1,n-1}$
 \item $\frac{n}{m+1}l\brak{m+1,n-1}$
 \item$ \frac{2^{n}}{m+1}+\frac{n}{m+1}l\brak{m+1,n-1}$
 \item $\frac{m}{n+1}l\brak{m+1,n-1}$
\end{enumerate}
\item If $f\brak{x}=\int_{x^{2}}^{x^{2}-1} e^{-t^{2}}dt$, then f\brak{x} increases in
\begin{enumerate}
    \item \brak{-2,2}
    \item no value of x
    \item \brak{0,\infty}
    \item \brak{-\infty,0}
    \hfill{{2003S}}
\end{enumerate}
  \item The area bounded by the curves $ y=\sqrt{x},2y+3=x$  and x-axis in the$ 1^{st}$ quadrant is
\begin{enumerate}
    \item 9
    \item $\frac{27}{4}$
    \item 36
    \item 18
    \hfill{{2003S}}
\end{enumerate}
 \item If f$\brak{x}$ is differentiable and $\int_{o}^{t^{2}}xf\brak{x}dx=\frac{2}{5}t^{5}$, then f$\brak{\frac{4}{25}}$
 \begin{enumerate}
	 \item $\frac{2}{5}$
	 \item $\frac{-5}{2}$
     \item 1
     \item $\frac{1}{2}$
     \hfill{{2004S}}
 \end{enumerate}
 \item The value of the integral $\int_{0}^{1}\sqrt{\frac{1-x}{1+x}dx}$ is
\begin{enumerate}
 \item $\frac{\pi}{2}$+1
 \item $\frac{\pi}{2}$-1
 \item -1
 \item 1
 \hfill{{2004S}}
 \end{enumerate}
 \item The area enclosed between the curves $y=ax^{2}$ and $x=ay^{2}$ and the line $y=1/4$ is
 \begin{enumerate}
	 \item $\frac{1}{\sqrt{3}}$
	 \item $\frac{1}{2}$
     \item 1
     \item $\frac{1}{3}$
     \hfill{{2004S}}
 \end{enumerate}
 \item $\int_{-2}^{0}\{x^{3}+3x^{2}+3x+3+\brak{x+1}\cos\brak{x+1}$\}dx
\begin{enumerate}
 \item -4
 \item 0
 \item 4
 \item 6
 \hfill{{2005S}}
\end{enumerate}
  \item The area bounded by the parabolas $y=\brak{x+1}^2$ and $y=\brak{x-1}^2$ and the line $y=\frac{1}{4}$ is
\begin{enumerate}
    \item 4 sq units
    \item $\frac{1}{6}$ sq units
    \item $\frac{4}{3}$ sq units
    \item $\frac{1}{3}$ sq units
    \hfill{{2005S}}
\end{enumerate}
 \item The area of the region between the curves $y=\sqrt{\frac{1+\sin x}{\cos x}}$ and $y=\sqrt{\frac{1-\sin x}{\cos x}}$ bounded by the lines $x=0$ and $x=\frac{\pi}{4}$ is
\begin{enumerate}
    \item $\int_{0}^{\sqrt{2}-1}\frac{t}{1+t^{2}\sqrt{1-t^{2}}dt}$
       \item $\int_{0}^{\sqrt{2}-1}\frac{4t}{1+t^{2}\sqrt{1-t^{2}}dt}$
      \item $\int_{0}^{\sqrt{2}+1}\frac{4t}{1+t^{2}\sqrt{1-t^{2}}dt}$ 
         \item $\int_{0}^{\sqrt{2}+1}\frac{t}{1+t^{2}\sqrt{1-t^{2}}dt}$
         \hfill{{2008S}}
\end{enumerate}
\item Let f be a non negative function defined on the interval $\sbrak{0,1}$. If $\int_{0}^{x}\sqrt{1-\brak{f'\brak{t}}^{2}}dt=\int_{0}^{x}f\brak{t}dt,0\leq{x}\leq1$,and $f\brak{0}=0,$ then
\begin{enumerate}
    \item$ f\brak{\frac{1}{2}}<\frac{1}{2}$ and $f\brak{\frac{1}{3}}>\frac{1}{3}$
    \item$ f\brak{\frac{1}{2}}>\frac{1}{2}$ and $f\brak{\frac{1}{3}}>\frac{1}{3}$
  \item$ f\brak{\frac{1}{2}}<\frac{1}{2}$ and $f\brak{\frac{1}{3}}<\frac{1}{3}$
  \item$ f\brak{\frac{1}{2}}>\frac{1}{2}$ and $f\brak{\frac{1}{3}}<\frac{1}{3}$
  \hfill{{2009S}}
\end{enumerate}
\end{enumerate}
\end{document}
