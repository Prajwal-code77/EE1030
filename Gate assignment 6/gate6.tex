\let\negmedspace\undefined
\let\negthickspace\undefined
\documentclass[journal]{IEEEtran}
\usepackage[a5paper, margin=10mm, onecolumn]{geometry}
%\usepackage{lmodern} % Ensure lmodern is loaded for pdflatex
\usepackage{tfrupee} % Include tfrupee package

\setlength{\headheight}{1cm} % Set the height of the header box
\setlength{\headsep}{0mm}     % Set the distance between the header box and the top of the text

\usepackage{gvv-book}
\usepackage{gvv}
\usepackage{cite}
\usepackage{amsmath,amssymb,amsfonts,amsthm}
\usepackage{algorithmic}
\usepackage{graphicx}
\usepackage{textcomp}
\usepackage{xcolor}
\usepackage{txfonts}
\usepackage{listings}
\usepackage{enumitem}
\usepackage{mathtools}
\usepackage{gensymb}
\usepackage{comment}
\usepackage[breaklinks=true]{hyperref}
\usepackage{tkz-euclide} 
\usepackage{listings}
% \usepackage{gvv}                                        
\def\inputGnumericTable{}                                 
\usepackage[latin1]{inputenc}                                
\usepackage{color}                                            
\usepackage{array}                                            
\usepackage{longtable}                                       
\usepackage{calc}                                             
\usepackage{multirow}                                         
\usepackage{hhline}                                           
\usepackage{ifthen}                                           
\usepackage{lscape}
\begin{document}

\bibliographystyle{IEEEtran}
\vspace{3cm}

\title{PH:PHYSICS-2013}
\author{AI24BTECH11005 - Bhukya Prajwal Naik
}
% \maketitle
% \newpage
% \bigskip
{\let\newpage\relax\maketitle}

\renewcommand{\thefigure}{\theenumi}
\renewcommand{\thetable}{\theenumi}
\setlength{\intextsep}{10pt} % Space between text and floats


\numberwithin{equation}{enumi}
\numberwithin{figure}{enumi}
\renewcommand{\thetable}{\theenumi}


\begin{enumerate}
    

 
%16
    \item If $\vec{A}$ and $\vec{B}$ are constant vectors, then $\nabla(\vec{A} \cdot \vec{B} \times \vec{r})$ is
   
   \begin{multicols}{4}
			\begin{enumerate}

\item $\vec{A} \cdot \vec{B}$
\item  $\vec{A} \times \vec{B}$
\item $\vec{r}$

\item Zero

        \end{enumerate}
		\end{multicols}
  \item  $\Gamma(n+\frac{1}{2})$ is equal to [Given $\Gamma(n+1)=n \Gamma(n)$ and $\left.\Gamma(1 / 2)=\sqrt{\pi}\right]$
   \begin{multicols}{4}
			\begin{enumerate}
\item  $\frac{n!}{2^{n}} \sqrt{\pi}$
\item $\frac{2 n!}{n!2^{n}} \sqrt{\pi}$
\item $\frac{2 n!}{n!2^{2 n}} \sqrt{\pi}$
\item $\frac{n!}{2^{2 n}} \sqrt{\pi}$

  \end{enumerate}
		\end{multicols}
  
  \item The relativistic form of Newton's second law of motion is
   \begin{multicols}{4}
			\begin{enumerate}
\item $F=\frac{m c}{\sqrt{c^{2}-v^{2}}} \frac{d v}{d t}$
\item $ F=\frac{m \sqrt{c^{2}-v^{2}}}{c} \frac{d v}{d t}$
\item  $F=\frac{m c^{2}}{c^{2}-v^{2}} \frac{d v}{d t}$
\item  $F=m \frac{{c}^{2}-{v}^{2}}{{c}^{2}} \frac{d v}{d t}$
   \end{enumerate}
		\end{multicols}
  \item Consider a gas of atoms obeying Maxwell-Boltzmann statistics. The average value of $e^{i \vec{a} \cdot \vec{p}}$ over all the momenta $\vec{p}$ of each of the particles (where $\vec{a}$ is a constant vector and $a$ is its magnitude, $m$ is the mass of each atom, $T$ is temperature and $k$ is Boltzmann's constant) is,
   \begin{multicols}{4}
			\begin{enumerate}
   \item  One
\item Zero
\item $e^{-\frac{1}{2} a^{2} m k T}$
\item $e^{-\frac{3}{2} a^{2} m k T}$
\end{enumerate}
		\end{multicols}
  \item The electromagnetic form factor $F(q^{2})$ of a nucleus is given by,
$$
F(q^{2})=\exp [-\frac{q^{2}}{2 Q^{2}}]
$$
where $Q$ is a constant. Given that
$$
\begin{gathered}
F(q^{2})=\frac{4 \pi}{q} \int_{0}^{\infty} r d r \rho(r) \sin q r \\
\int d^{3} r \rho(r)=1
\end{gathered}
$$
where $\rho(r)$ is the charge density, the root mean square radius of the nucleus is given by,
  Primeval
   \begin{multicols}{4}
			\begin{enumerate}
   \item $\frac{1} { Q}$
\item $\frac{\sqrt{2}}  {Q}$
\item$\frac{\sqrt{3}} { Q}$
\item $\frac{\sqrt{6}} { Q}$
\end{enumerate}
		\end{multicols}
  \item  A uniform circular disk of radius $R$ and mass $M$ is rotating with angular speed $\omega$ about an axis, passing through its center and inclined at an angle 60 degrees with respect to its symmetry axis. The magnitude of the angular momentum of the disk is,
  \begin{multicols}{4}
			\begin{enumerate}
\item $\frac{\sqrt{3}}{4} \omega M R^{2}$
\item $\frac{\sqrt{3}}{8} \omega M R^{2}$
\item $\frac{\sqrt{7}}{8} \omega M R^{2}$
\item $\frac{\sqrt{7}}{4} \omega M R^{2}$
\end{enumerate}
		\end{multicols}
  \item Consider two small blocks, each of mass $M$, attached to two identical springs. One of the springs is attached to the wall, as shown in the figure. The spring constant of each spring is $k$. The masses slide along the surface and the friction is negligible. The frequency of one of the normal modes of the system is,
  \begin{figure}[!ht]
\centering
\resizebox{1\textwidth}{!}{%
\begin{circuitikz}
\tikzstyle{every node}=[font=\LARGE]
\draw (9,13.75) to[short] (12.25,13.75);
\draw (12.25,13.75) to[short] (12.25,12.25);
\draw (12.25,12.25) to[short] (14,12.25);
\draw (14,12.25) to[short] (14,16.25);
\draw (14,16.25) to[short] (6.25,16.25);
\draw (6.25,16.25) to[short] (6.25,8.75);
\draw (6.25,8.75) to[short] (14.25,8.75);
\draw (12.25,11.75) to[short] (14.25,11.75);
\draw (12.25,11.75) to[short] (12.25,10.5);
\draw (12.25,10.5) to[short] (8.75,10.5);
\draw (9,13.75) to[short] (11,13.75);
\draw (9,13.75) to[short] (9,10.5);
\draw (14.25,11.75) to[short] (14.25,8.75);
\draw [ dashed] (7.75,14.5) rectangle  (13,9.75);
\draw [->, >=Stealth] (5.25,13.25) -- (5.75,13.25);
\draw [short] (5.75,13.25) -- (9.5,13.25);
\draw (6,12.75) to[short] (9.75,12.75);
\draw (6,12.25) to[short] (9.75,12.25);
\draw (6,11.75) to[short] (9.75,11.75);
\draw (6,11.25) to[short] (10,11.25);
\draw (6,10.75) to[short] (9.75,10.75);
\draw [->, >=Stealth] (6.25,9.75) -- (5.5,9.75);
\draw (5.5,9.75) to[short, -o] (4.75,9.75) ;
\draw (5.25,13.25) to[short, -o] (4.75,13.25) ;
\node [font=\LARGE] at (5.25,14) {I};
\draw [->, >=Stealth] (14.75,13.25) -- (14.75,12.25);
\draw [->, >=Stealth] (14.75,11) -- (14.75,11.75);
\node [font=\LARGE] at (15.5,12) {0.2 cm};
\draw (7.5,18) to[short] (7.5,17);
\draw (12.75,18) to[short] (12.75,17);
\draw [<->, >=Stealth] (7.5,17.5) -- (12.75,17.5);
\node [font=\LARGE] at (10,18.5) {10 cm};
\draw (3,14.75) to[short] (4.25,14.75);
\draw (3,9.75) to[short] (3.75,9.75);
\draw (3,9.75) to[short] (4.25,9.75);
\draw [<->, >=Stealth] (3.75,14.75) -- (3.75,9.75);
\node [font=\LARGE] at (2.5,12.25) {10 cm};
\end{circuitikz}
}%

\label{fig:my_label}
\end{figure}

   \begin{multicols}{4}
			\begin{enumerate}
  \item$\sqrt{\frac{3+\sqrt{2}}{2}} \sqrt{\frac{k}{M}}$
\item $\sqrt{\frac{3+\sqrt{3}}{2}} \sqrt{\frac{k}{M}}$
\item$\sqrt{\frac{3+\sqrt{5}}{2}} \sqrt{\frac{k}{M}}$
\item $\sqrt{\frac{3+\sqrt{6}}{2}} \sqrt{\frac{k}{M}}$
\end{enumerate}
		\end{multicols}
  
\item    A charge distribution has the charge density given by $\rho=Q\{\delta(x-x_{0})-\delta(x+x_{0})\}$. For this charge distribution the electric field at $(2 x_{0}, 0,0)$
\begin{multicols}{4}
			\begin{enumerate}

\item  $\frac{2 Q \hat{x}}{9 \pi \epsilon_{0} x_{0}^{2}}$
\item  $\frac{Q \hat{x}}{4 \pi \epsilon_{0} x_{0}^{3}}$
\item  $\frac{{Q} \hat{x}}{4 \pi \epsilon_{0} x_{0}^{2}}$
\item  $\frac{{Q} \hat{{x}}}{16 \pi \epsilon_{0} x_{0}^{2}}$
   \end{enumerate}
		\end{multicols}
  \item   A monochromatic plane wave at oblique incidence undergoes reflection at a dielectric interface. If $\hat{k}_{i}, \hat{k}_{r}$ and $\hat{n}$ are the unit vectors in the directions of incident wave, reflected wave and the normal to the surface respectively, which one of the following expressions is correct?
  \begin{multicols}{4}
			\begin{enumerate}
   \item$(\hat{k}_{i}-\hat{k}_{r}) \times \hat{n} \neq 0$
\item $(\hat{k}_{i}-\hat{k}_{r}) \cdot \hat{n}=0$
\item  $(\hat{k}_{i} \times \hat{n}) \cdot \hat{k}_{r}=0$
\item $(\hat{k}_{i} \times \hat{n}) \cdot \hat{k}_{r} \neq 0$
 \end{enumerate}
		\end{multicols}
  \item  In a normal Zeeman effect experiment, spectral splitting of the line at the wavelength 643.8 nm corresponding to the transition $5{ }^{1} D_{2} \rightarrow 5{ }^{1} P_{1}$ of cadmium atoms is to be observed. The spectrometer has a resolution of 0.01 nm . The minimum magnetic field needed to observe this is $(m_{e}=9.1 \times 10^{-31} {~kg}, e=1.6 \times 10^{-19} {C}, c=3 \times 10^{8} \mathrm{~m} / \mathrm{s}.$ )
 
			\begin{enumerate}
   \item $0.26 T$
\item  $0.52 T$
\item $2.6 T$
\item$5.2 T$
 \end{enumerate}
	
  \item The spacing between vibrational energy levels in CO molecule is found to be $8.44 \times 10^{-2} \mathrm{eV}$. Given that the reduced mass of CO is $1.14 \times 10^{-26} \mathrm{~kg}$, Planck's constant is $6.626 \times 10^{-34} \mathrm{Js}$ and $1 \mathrm{eV}=1.6 \times 10^{-19} \mathrm{~J}$. The force constant of the bond in CO molecule is
  

\begin{multicols}{4}
			\begin{enumerate}
   \item $1.87 {~N} / {m}$
\item $18.7 {~N} / {m}$
\item $187 {~N} / {m}$
\item $1870 {~N} / {m}$
\end{enumerate}
		\end{multicols}
  \item   A lattice has the following primitive vectors (in $\AA$ ): $\vec{a}=2(\hat{\hat j}+\hat{k}), \quad \vec{b}=2(\hat{k}+\hat{i}), \quad \vec{c}=2(\hat{i}+\hat{j})$. The reciprocal lattice corresponding to the above lattice is
 
			\begin{enumerate}
   \item BCC lattice with cube edge of $(\frac{\pi}{2}) \AA^{-1}$
\item BCC lattice with cube edge of $(2 \pi) \AA^{-1}$
\item FCC lattice with cube edge of $(\frac{\pi}{2}) \AA^{-1}$
\item FCC lattice with cube edge of $(2 \pi) \AA^{-1}$
  \end{enumerate}
		
 \end{enumerate}

\end{document}
