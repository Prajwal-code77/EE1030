\let\negmedspace\undefined
\let\negthickspace\undefined
\documentclass[journal]{IEEEtran}
\usepackage[a5paper, margin=10mm, onecolumn]{geometry}
%\usepackage{lmodern} % Ensure lmodern is loaded for pdflatex
\usepackage{tfrupee} % Include tfrupee package

\setlength{\headheight}{1cm} % Set the height of the header box
\setlength{\headsep}{0mm}     % Set the distance between the header box and the top of the text

\usepackage{gvv-book}
\usepackage{gvv}
\usepackage{cite}
\usepackage{amsmath,amssymb,amsfonts,amsthm}
\usepackage{algorithmic}
\usepackage{graphicx}
\usepackage{textcomp}
\usepackage{xcolor}
\usepackage{txfonts}
\usepackage{listings}
\usepackage{enumitem}
\usepackage{mathtools}
\usepackage{gensymb}
\usepackage{comment}
\usepackage[breaklinks=true]{hyperref}
\usepackage{tkz-euclide} 
\usepackage{listings}
% \usepackage{gvv}                                        
\def\inputGnumericTable{}                                 
\usepackage[latin1]{inputenc}                                
\usepackage{color}                                            
\usepackage{array}                                            
\usepackage{longtable}                                       
\usepackage{calc}                                             
\usepackage{multirow}                                         
\usepackage{hhline}                                           
\usepackage{ifthen}                                           
\usepackage{lscape}
\begin{document}

\bibliographystyle{IEEEtran}
\vspace{3cm}

\title{MA:MATHEMATICS-2013}
\author{AI24BTECH11005 - Bhukya Prajwal Naik
}
% \maketitle
% \newpage
% \bigskip
{\let\newpage\relax\maketitle}

\renewcommand{\thefigure}{\theenumi}
\renewcommand{\thetable}{\theenumi}
\setlength{\intextsep}{10pt} % Space between text and floats


\numberwithin{equation}{enumi}
\numberwithin{figure}{enumi}
\renewcommand{\thetable}{\theenumi}


\begin{enumerate}
    

 
%16
    \item Let $q_{1}, \ldots, q_{n_{0}+1}$ be $n_{0}+1$ distinct points and $Y=X \backslash\{q_{1}, \ldots, q_{n_{0}+1}\}$. Let $m$ be the number of connected components of $Y$. The maximum possible value of $m$ is $\qquad$
    \\
    \section*{Statement for Linked Answer Questions 54 and 55:}Let $W\left(y_{1}, y_{2}\right)$ be the Wronskian of two linearly independent solutions $y_{1}$ and $y_{2}$ of the equation $y^{\prime \prime}+$ $P(x) y^{\prime}+Q(x) y=0$.
    
  \item The product $W(y_{1}, y_{2}) P(x)$ equals
   \begin{multicols}{4}
			\begin{enumerate}

\item  $y_{2} y_{1}^{\prime \prime}-y_{1} y_{2}^{\prime \prime}$
\item  $y_{1} y_{2}^{\prime}-y_{2} y_{1}^{\prime}$
\item $y_{1}^{\prime} y_{2}^{\prime \prime}-y_{2}^{\prime} y_{1}^{\prime \prime}$

\item $y_{2}^{\prime} y_{1}^{\prime}-y_{1}^{\prime \prime} y_{2}^{\prime \prime}$

        \end{enumerate}
		\end{multicols}
  \item  If $y_{1}=e^{2 x}$ and $y_{2}=x e^{2 x}$, then the value of $P(0)$ is
   \begin{multicols}{4}
			\begin{enumerate}
\item  4
\item -4
\item 2
\item -2

  \end{enumerate}
		\end{multicols}
  \section*{General Aptitude (GA) Questions}
  \section*{Q. 56 - Q. 60 carry one mark each.}
  \item  A number is as much greater than 75 as it is smaller than 117. The number is:
   \begin{multicols}{4}
			\begin{enumerate}
\item 91
\item  93
\item89
\item 96
   \end{enumerate}
		\end{multicols}
  \item The professor ordered to the students to go out of the class.
Which of the given sentence is gramatically wrong

   \begin{multicols}{4}
			\begin{enumerate}
   \item  The professor
\item ordered  to
\item the students
\item of the class
\end{enumerate}
		\end{multicols}
  \item Which of the following options is the closest in meaning to the word given below:\\
  Primeval
   \begin{multicols}{4}
			\begin{enumerate}
   \item  Modern
\item Historic
\item Primitive 
\item Antique
\end{enumerate}
		\end{multicols}
  \item Friendship, no matter how ($\qquad)$ it is, has its limitations
  \begin{multicols}{4}
			\begin{enumerate}
\item cordial
\item intimate
\item secret
\item pleasant
\end{enumerate}
		\end{multicols}
  \item Select the pair that best expresses a relationship similar to that expressed in the pair: Medicine: Health
   \begin{multicols}{1}
			\begin{enumerate}
  \item Science: Experiment
\item Wealth: Peace
\item Education: Knowledge
\item Happiness 
\end{enumerate}
		\end{multicols}
  \section*{Q. 61 to Q. 65 carry two marks each.}
\item   X and Y are two positive real numbers such that $2 X+Y \leq 6$ and $X+2 Y \leq 8$. For which of the following values of $(X, Y)$ the function $f(X, Y)=3 X+6 Y$ will give maximum value?
\begin{multicols}{4}
			\begin{enumerate}

\item  $(\frac{4}  {3},\frac{10} { 3})$
\item $(\frac{8} {3},\frac{20} {3})$
\item $(\frac{8} {3},\frac{10}{ 3})$
\item $(\frac{4} { 3},\frac{20} { 3})$
   \end{enumerate}
		\end{multicols}
  \item  If $|4 X-7|=5$ then the values of $2|X|-|-X|$ is:
  \begin{multicols}{4}
			\begin{enumerate}
   \item $2,\frac{1} {3}$
\item 1.0
\item  2.0
\item 4.0
 \end{enumerate}
		\end{multicols}
  \item Following table provides figures (in rupees) on annual expenditure of a firm for two years - 2010 and 2011.\\
\\In 2011, which of the following two categories have registered increase by same percentage?
\begin{table}[h!]
	\centering
	 \begin{tabular}{|l|c|c|}
\hline \multicolumn{1}{|c|}{ Category } & $\mathbf{2 0 1 0}$ & $\mathbf{2 0 1 1}$ \\
\hline Raw material & 5200 & 6240 \\
\hline Power \& fuel & 7000 & 9450 \\
\hline Salary \& wages & 9000 & 12600 \\
\hline Plant \& machinery & 20000 & 25000 \\
\hline Advertising & 15000 & 19500 \\
\hline Research \& Development & 22000 & 26400 \\
\hline
\end{tabular}


\end{table}
 
			\begin{enumerate}
   \item Raw material and Salary \& wages
\item Salary \& wages and Advertising
\item  Power \& fuel and Advertising
\item Raw material and Research \& Development
 \end{enumerate}
	
  \item   A firm is selling its product at Rs. 60 per unit. The total cost of production is Rs. 100 and firm is earning total profit of Rs. 500. Later, the total cost increased by $30 \%$. By what percentage the price should be increased to maintained the same profit level.
  

\begin{multicols}{4}
			\begin{enumerate}
   \item 5
\item 10
\item15
\item 30
\end{enumerate}
		\end{multicols}
  \item   Abhishek is elder to Savar.

Savar is younger to Anshul.
Which of the given conclusions is logically valid and is inferred from the above statements?
 
			\begin{enumerate}
   \item Abhishek is elder to Anshul
\item Anshul is elder to Abhishek
\item  Abhishek and Anshul are of the same age
\item No conclusion follows
  \end{enumerate}
		
 \end{enumerate}

\end{document}
