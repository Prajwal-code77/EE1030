\let\negmedspace\undefined
\let\negthickspace\undefined
\documentclass[journal]{IEEEtran}
\usepackage[a5paper, margin=8mm, onecolumn]{geometry}
%\usepackage{lmodern} % Ensure lmodern is loaded for pdflatex
\usepackage{tfrupee} % Include tfrupee package

\setlength{\headheight}{1cm} % Set the height of the header box
\setlength{\headsep}{0mm}     % Set the distance between the header box and the top of the text

\usepackage{gvv-book}
\usepackage{gvv}
\usepackage{cite}
\usepackage{amsmath,amssymb,amsfonts,amsthm}
\usepackage{algorithmic}
\usepackage{graphicx}
\usepackage{textcomp}
\usepackage{xcolor}
\usepackage{txfonts}
\usepackage{listings}
\usepackage{enumitem}
\usepackage{mathtools}
\usepackage{gensymb}
\usepackage{comment}
\usepackage[breaklinks=true]{hyperref}
\usepackage{tkz-euclide} 
\usepackage{listings}
% \usepackage{gvv}                                        
\def\inputGnumericTable{}                                 
\usepackage[latin1]{inputenc}                                
\usepackage{color}                                            
\usepackage{array}                                            
\usepackage{longtable}                                       
\usepackage{calc}                                             
\usepackage{multirow}                                         
\usepackage{hhline}                                           
\usepackage{ifthen}                                           
\usepackage{lscape}
\begin{document}

\bibliographystyle{IEEEtran}
\vspace{3cm}

\title{2024-April-5-Shift-2}
\author{AI24BTECH11005 - Prajwal Naik}
% \maketitle
% \newpage
% \bigskip
{\let\newpage\relax\maketitle}

\renewcommand{\thefigure}{\theenumi}
\renewcommand{\thetable}{\theenumi}
\setlength{\intextsep}{10pt} % Space between text and floats


\numberwithin{equation}{enumi}
\numberwithin{figure}{enumi}
\renewcommand{\thetable}{\theenumi}

\begin{enumerate}
\setcounter{enumi}{0}
 
%16
    \item 60 words can be made using all the letters of the word BHBJO with or without meaning. If these words are written as in a dictionary , then the $50^{th}$ word is:
    \item{[Apr 2024]}
        \begin{multicols}{4}
            \begin{enumerate}
                \item OBBJH
                \item JBBOH
                \item HBBJO
                \item OBBHJ
            \end{enumerate}
        \end{multicols}

%17
    \item Let the $S=\{2,4,8,16.....,512\}$ be partioned into 3 sets , A,B,C with equal number of elements such that $AUBUC=S$, and $A \bigcap B=B \bigcap C=C \bigcap A=\phi$. The maximum number of such possible partitions of S is equal to
     \item{[Apr 2024]}


		\begin{multicols}{4}
			\begin{enumerate}
				\item 1680
				\item  1520
				\item 1710
				\item  1640
			\end{enumerate}
		\end{multicols}

%18
    \item The area enclosed between the curves $y=x\abs{x}$ and $y=x-\abs{x}$ is  :
     \item{[Apr 2024]}
        \begin{multicols}{4}
            \begin{enumerate}
              \item 1
              \item $\frac{8}{3}$
              \item $\frac{4}{3}$
              \item $\frac{2}{3}$
            \end{enumerate}
        \end{multicols}

%19
    \item If the constant term in the expansion of $(\frac{3^{0.2}}{x}+\frac{2x}{5^\frac{1}{3}})^{12}$, $x \neq 0$, $\alpha .2^{8}.3^{0.2}$, then $25\alpha$ is  equal to :
     \item{[Apr 2024]}
		\begin{multicols}{4}
			\begin{enumerate}
				\item 639
    \item 693
    \item 742
    \item 724
			\end{enumerate}
		\end{multicols}

%20 
    \item Consider three vectors $\overrightarrow{a},\overrightarrow{b},\overrightarrow{c}$, Let $\overrightarrow{a}=2,\overrightarrow{b}=3$ and $\overrightarrow{a}=\overrightarrow{b}X\overrightarrow{c}$, $\alpha \in [0,\frac{\pi}{3}]$, is the angle between the vectors $\overrightarrow{b}$ and $\overrightarrow{c}$, then the minimum value of $\abs{\overrightarrow{c}-\overrightarrow{a}}^2$ is equal to:
     \item{[Apr 2024]}
		\begin{multicols}{4}
			\begin{enumerate}
				
				\item 124
    \item 110
     \item 105
      \item 121
			\end{enumerate}
		\end{multicols}

%21
    \item Let $\beta(m,n)=\int_{0}^{1} x^{m-1} (1-x)^{n-1}dx, m,n\geq0$, If $\int_{0}^{1} (1-x^10)^20 dx=a.\beta(b,c)$, then 100(a+b+c) equals:
     \item{[Apr 2024]}
    \begin{multicols}{4}
            \begin{enumerate}
              \item 1021
              \item 1120
              \item 2012
              \item 2120
            \end{enumerate}
        \end{multicols}
%22
    \item Let $\brak{\alpha,\beta,\gamma}$ be the image of the point \brak{8,5,7} in the line $\frac{x-1}{2}=\frac{y+1}{3}=\frac{z-2}{5}$, then $\alpha+\beta+\gamma$ is equal to : 
     \item{[Apr 2024]}
    \begin{multicols}{4}
            \begin{enumerate}
              \item 18
              \item  16
              \item  20
              \item 14 
              \end{enumerate}
        \end{multicols}

%23
    \item For $x\geq 0$, the least value of K, for which $4^{1+x} +4^{1-x}, \frac{k}{2}, 16^x +16^{-x}$, are three consecutive terms of an AP , is equal to
     \item{[Apr 2024]}
\begin{multicols}{4}
            \begin{enumerate}
              \item 4
              \item 10
              \item 16
              \item 8
            \end{enumerate}
        \end{multicols}
%24
    \item $y\brak{\theta}=\frac{2\cos{\theta}+\cos{2\theta}}{\cos{3\theta}+4\cos{2\theta}+5\cos{\theta}+2}$, then at $\theta=\frac{\pi}{2}$, $y''+y'+y$ is equal to :
     \item{[Apr 2024]}
    \begin{multicols}{4}
            \begin{enumerate}
              \item $\frac{1}{2}$
              \item $\frac{3}{2}$
              \item $2$
              \item 1
            \end{enumerate}
        \end{multicols}
%25
    \item Let $S_1=\cbrak{z \in C: \abs{z}\leq 5}$, $S_2=\cbrak{z \in C :Im(\frac{z+1-\sqrt{3}i}{1-\sqrt{3}i})}$
     \item{[Apr 2024]}
    \begin{multicols}{4}
            \begin{enumerate}
              \item $\frac{125\pi}{24}$
              \item $\frac{125\pi}{6}$
              \item $\frac{125\pi}{12}$
              \item $\frac{125\pi}{4}$
            \end{enumerate}
        \end{multicols}
     
    \item Let the circle $C_1: x^2+y^2-2(x+y)+1=0$, and $C_2$ be a circle having centre at $(-1,0)$ and radius 2 . If the line of common chord of $C1 \text{and } C_2$ intersects the y-axis at the point P , then the square of the distance of P from the centre $C_1$ is :
     \item{[Apr 2024]}
    \begin{multicols}{4}
            \begin{enumerate}
              \item 6
              \item 2
              \item 4
              \item 1
            \end{enumerate}
        \end{multicols}
    
    

    \item The differential equation of the family of circles passing through the origin and having centre at the line $y=x$ is:
     \item{[Apr 2024]}
    \begin{multicols}{2}
            \begin{enumerate}
             \item $(x^2-y^2+2xy)dx=(x^2-y^2-2xy)dy$
              \item $(x^2+y^2+2xy)dx=(x^2+y^2-2xy)dy$
               \item $(x^2+y^2-2xy)dx=(x^2+y^+2xy)dy$
                \item $(x^2-y^2+2xy)dx=(x^2-y^2+2xy)dy$
            \end{enumerate}
        \end{multicols}
    \item $f:[-1,2]\rightarrow R$, be given by $f(x)=2x^2+x+[x^2]-[x]$, The number of points , where f is  not continuous is:
     \item{[Apr 2024]}
    \begin{multicols}{4}
            \begin{enumerate}
        
              \item 19
              \item 38
              \item 57
              \item 76
            \end{enumerate}
        \end{multicols}
    
    \item let $f,g: R\rightarrow R $ be defined as : $f(x)=\abs{x-1}$ and $g(x)=e^x, when x\geq 0$,$g(x)=x+1, x\leq 0$ Then $f(g(x))$ is
     \item{[Apr 2024]}
    \begin{multicols}{2}
            \begin{enumerate}
              \item neither one-one nor onto
              \item both one-one and onto
              \item one-one not onto
              \item onto but not one-one
            \end{enumerate}
        \end{multicols}
        \item The coefficients a,b,c in the quadratic equation $ax^2+bx+c=0$ are from the set $\cbrak{1,2,3,4,5,6}$ . If the probability of this equation having one root bigger than the other is p, then 216p is equal to:
         \item{[Apr 2024]}
        \begin{multicols}{4}
            \begin{enumerate}
              \item 19
              \item  38
              \item 57
              \item 76
            \end{enumerate}
        \end{multicols}
 \end{enumerate}

\end{document}
