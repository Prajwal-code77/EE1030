\let\negmedspace\undefined
\let\negthickspace\undefined
\documentclass[journal]{IEEEtran}
\usepackage[a5paper, margin=8mm, onecolumn]{geometry}
%\usepackage{lmodern} % Ensure lmodern is loaded for pdflatex
\usepackage{tfrupee} % Include tfrupee package

\setlength{\headheight}{1cm} % Set the height of the header box
\setlength{\headsep}{0mm}     % Set the distance between the header box and the top of the text

\usepackage{gvv-book}
\usepackage{gvv}
\usepackage{cite}
\usepackage{amsmath,amssymb,amsfonts,amsthm}
\usepackage{algorithmic}
\usepackage{graphicx}
\usepackage{textcomp}
\usepackage{xcolor}
\usepackage{txfonts}
\usepackage{listings}
\usepackage{enumitem}
\usepackage{mathtools}
\usepackage{gensymb}
\usepackage{comment}
\usepackage[breaklinks=true]{hyperref}
\usepackage{tkz-euclide} 
\usepackage{listings}
% \usepackage{gvv}                                        
\def\inputGnumericTable{}                                 
\usepackage[latin1]{inputenc}                                
\usepackage{color}                                            
\usepackage{array}                                            
\usepackage{longtable}                                       
\usepackage{calc}                                             
\usepackage{multirow}                                         
\usepackage{hhline}                                           
\usepackage{ifthen}                                           
\usepackage{lscape}
\begin{document}

\bibliographystyle{IEEEtran}
\vspace{3cm}

\title{Mechanical Engineering-2007}
\author{AI24BTECH11005 - Prajwal Naik}
% \maketitle
% \newpage
% \bigskip
{\let\newpage\relax\maketitle}

\renewcommand{\thefigure}{\theenumi}
\renewcommand{\thetable}{\theenumi}
\setlength{\intextsep}{10pt} % Space between text and floats


\numberwithin{equation}{enumi}
\numberwithin{figure}{enumi}
\renewcommand{\thetable}{\theenumi}

\begin{enumerate}
\setcounter{enumi}{17}
 

    \item In orthogonal turning of a low carbon steel bar of diameter 150 mm with uncoated carbide tool, the cutting velocity is 90 m/min. The feed is 0.24 mm/rev and the depth of cut is 2 mm . The chip thickness obtained is 0.48 mm . If the orthogonal rake angle is zero and the principal cutting edge angle is $90^{\circ}$, the shear angle in degree is
        \begin{multicols}{4}
            \begin{enumerate}
                \item 20.56
                \item 26.56
                \item 30.56
                \item 36.56
            \end{enumerate}
        \end{multicols}


    \item Which type of motor is NOT used in axis or spindle drives of CNC machine tools?


		\begin{multicols}{1}
			\begin{enumerate}
	\item induction motor
    \item de servo motor
    \item stepper motor
    \item linear servo motor
			\end{enumerate}
		\end{multicols}


    \item Volume of a cube of side ' $l$ ' and volume of a sphere of radius ' $r$ ' are equal. Both the cube and the sphere are solid and of same material. They are being cast. The ratio of the solidification time of the cube to the same of the sphere is
        \begin{multicols}{4}
            \begin{enumerate}
              \item $\brak{\frac{4 \pi}{6}}^{3}\brak{\frac{r}{l}}^{6}$
              \item $\brak{\frac{4 \pi}{6}}\brak{\frac{r}{l}}^{2}$
              \item $\brak{\frac{4 \pi}{6}}^{2}\brak{\frac{r}{l}}^{3}$
              \item $\brak{\frac{4 \pi}{6}}^{2}\brak{\frac{r}{l}}^{4}$
            \end{enumerate}
        \end{multicols}


    \item If $y=x+\sqrt{x+\sqrt{x+\sqrt{x+.... \infty}}}$, then $y(2)=$
		\begin{multicols}{4}
			\begin{enumerate}
	\item 4 or 1 
    \item 4 only
    \item 1 only
    \item undefined
	\end{enumerate}
		\end{multicols}

 
    \item The area of a triangle formed by the tips of vectors $\overrightarrow{a}, \overrightarrow{b}$ and $\overrightarrow{c}$ is
		\begin{multicols}{1}
			\begin{enumerate}
				
	\item	$\frac{1}{2}\brak{\abs{\brak{\overrightarrow{a}-\overrightarrow{b}}.\brak{\overrightarrow{a}-\overrightarrow{c}}}}$	
    \item $\frac{1}{2}\brak{\abs{\overrightarrow{a}-\overrightarrow{b})X(\overrightarrow{a}-\overrightarrow{c}}}$	
     \item $\frac{1}{2}\abs{\overrightarrow{a}X\overrightarrow{b}X\overrightarrow{c}}$
      \item $\frac{1}{2}\abs{\overrightarrow{a}X\overrightarrow{b}.\overrightarrow{c}}$
			\end{enumerate}
		\end{multicols}


    \item The solution of $\frac{d y}{d x}=y^{2}$ with initial value $y(0)=1$ is bounded in the interval
    \begin{multicols}{4}
            \begin{enumerate}
        \item $-\infty \leq x \leq \infty$
        \item $-\infty \leq x \leq 1$    
        \item $x<1, x>1$
        \item $-2 \leq x \leq 2$
        \end{enumerate}
        \end{multicols}

    \item If $F(s)$ is the Laplace transform of function $f(t)$, then Laplace transform of $\int_{0}^{t} f(\tau) d \tau$ is
    \begin{multicols}{4}
            \begin{enumerate}
              \item  $\frac{1}{s} F(s)$
              \item $\frac{1}{s} F(s)-f(0)$
              \item $s F(s)-f(0)$
              \item  $\int F(s) d s$
              \end{enumerate}
        \end{multicols}


    \item A calculator has accuracy up to 8 digits after decimal place. The value of $\int_{0}^{2 \pi} \sin x d x$ when evaluated using this calculator by trapezoidal method with 8 equal intervals, to 5 significant digits is
\begin{multicols}{4}
 \begin{enumerate}
\item 0.00000
\item 1.0000
\item 0.00500
\item 0.00025
            \end{enumerate}
        \end{multicols}

    \item Let $X$ and $Y$ be two independent random variables. Which one of the relations between expectation (E), variance (Var) and covariance (Cov) given below is FALSE?
    \begin{multicols}{1}
            \begin{enumerate}
        \item $E(X Y)=E(X) E(Y)$
         \item $Cov(X,Y)=0$
         \item $Var(X+Y)=Var(X)+Var(Y)$
         \item $E(X^2Y^2)=(E(X))^2((E(Y))^2)$
            \end{enumerate}
        \end{multicols}

\item $\lim _{x \rightarrow 0} \frac {e^{x}-1+x+\frac{x^{2}}{2}}{x^{3}}=$
    \begin{multicols}{4}
            \begin{enumerate}
              \item 0
              \item $\frac{1}{6}$
              \item $\frac{1}{3}$
              \item 1
            \end{enumerate}
        \end{multicols}
     
    \item The number of linearly independent eigenvectors of \myvec{2 &1 \\ 0 & 2} is
    \begin{multicols}{4}
            \begin{enumerate}
              \item 0
              \item 1
              \item 2
              \item infinite
            \end{enumerate}
        \end{multicols}
    
    

    \item The inlet angle of runner blades of a Francis turbine is $90^{\circ}$. The blades are so shaped that the tangential component of velocity at blade outlet is zero. The flow velocity remains constant throughout the blade passage and is equal to half of the blade velocity at runner inlet. The blade efficiency of the runner is
    \begin{multicols}{4}
            \begin{enumerate}
            \item $25 \%$
            \item $50 \%$
            \item $80 \%$
            \item $89 \%$
            \end{enumerate}
        \end{multicols}
    \item  The temperature distribution within the thermal boundary layer over a heated isothermal flat plate is given by
	    $\frac{T-T_w}{T_\infty-T_w}=\frac{3}{2}(\frac{y}{\delta_t})-\frac{1}{2}(\frac{y}{\delta_t})^3$, where $T_w$ and $T_\infty$ are the temperatures of plate and free stream respectively, and y is the normal distance measured from the plate. The local Nusselt number based on the thermal boundary layer thickness $\delta_{t}$ is given by
    \begin{multicols}{4}
            \begin{enumerate}
              \item 1.33
              \item 1.50
              \item 2.0
              \item 4.64
            \end{enumerate}
        \end{multicols}
    \item In a counterflow heat exchanger , hot fluid enters at $60^{\circ} C$ and cold fluid leaves at $30^{\circ} C$ Mass flow rate of the hot fluid is 1 kg/s and that of the cold fluid is 2 kg/s . Specific heat of the hot fluid is 10 kJ/KgK and that of the cold  fluid is 5 kJ/KgK. The Log Mean Temperature for the heat exchanger in ${\circ} C$ is 
    \begin{multicols}{4}
            \begin{enumerate}
              \item 15
              \item 30
              \item 35
              \item 45
            \end{enumerate}
        \end{multicols}
        \item The average heat transfer coefficient on a thin hot vertical plate suspended in still air can be determined from observations of the change in plate temperature with time as it cools. Assume the plate temperature to be uniform at any instant of time and radiation heat exchange with the surroundings negligible. The ambient temperature is $25{\circ} C $  the plate has a total surface area of 0.1$m^2$ and a mass of 4 kg . The specific heat of the plate material is 2.5 kJ/kgK The convective heat transfer coefficient in W/$m^2$K at the instant when the plate temperature is $225{\circ}C$ and the change in plate temperature with time $\frac{dT}{dt}=-0.02K/s$, is
        \begin{multicols}{4}
            \begin{enumerate}
              \item 200
              \item  20
              \item 15
              \item 10
            \end{enumerate}
        \end{multicols}
\item A model of a hydraulic turbine is tested at a head of $\frac{1}{4}^{th}$ of that under which the full scale turbine works. The diameter of the model is half of that of the full scale turbine. If N is the RPM of the full scale turbine, then the RPM of the model will be
        \begin{multicols}{4}
            \begin{enumerate}
              \item $\frac{N}{4}$
              \item  $\frac{N}{2}$
              \item N
              \item 2N
            \end{enumerate}
        \end{multicols}
        \item  The stroke and bore of a four stroke spark ignition engine are 250 mm and 200 mm respectively. The clearance volume is 0.001 $m^3$. If the specific heat ratio $\gamma=1.4$,
        the air-standard cycle efficiency of the engine is
        \begin{multicols}{4}
        \begin{enumerate}
        \item $46.40 \%$
        \item $56.10 \%$
          \item  $58.20 \%$
          \item $62.80 \%$
          \end{enumerate}
        \end{multicols}

        
 \end{enumerate}

\end{document}
