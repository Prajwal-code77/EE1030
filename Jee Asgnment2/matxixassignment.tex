\let\negmedspace\undefined
\let\negthickspace\undefined
\documentclass[journal]{IEEEtran}
\usepackage[a5paper, margin=8mm, onecolumn]{geometry}
%\usepackage{lmodern} % Ensure lmodern is loaded for pdflatex
\usepackage{tfrupee} % Include tfrupee package

\setlength{\headheight}{1cm} % Set the height of the header box
\setlength{\headsep}{0mm}     % Set the distance between the header box and the top of the text

\usepackage{gvv-book}
\usepackage{gvv}
\usepackage{cite}
\usepackage{amsmath,amssymb,amsfonts,amsthm}
\usepackage{algorithmic}
\usepackage{graphicx}
\usepackage{textcomp}
\usepackage{xcolor}
\usepackage{txfonts}
\usepackage{listings}
\usepackage{enumitem}
\usepackage{mathtools}
\usepackage{gensymb}
\usepackage{comment}
\usepackage[breaklinks=true]{hyperref}
\usepackage{tkz-euclide} 
\usepackage{listings}
% \usepackage{gvv}                                        
\def\inputGnumericTable{}                                 
\usepackage[latin1]{inputenc}                                
\usepackage{color}                                            
\usepackage{array}                                            
\usepackage{longtable}                                       
\usepackage{calc}                                             
\usepackage{multirow}                                         
\usepackage{hhline}                                           
\usepackage{ifthen}                                           
\usepackage{lscape}
\begin{document}

\bibliographystyle{IEEEtran}
\vspace{3cm}

\title{2022-July-27 Shift-2}
\author{AI24BTECH11005 - Prajwal Naik}
% \maketitle
% \newpage
% \bigskip
{\let\newpage\relax\maketitle}

\renewcommand{\thefigure}{\theenumi}
\renewcommand{\thetable}{\theenumi}
\setlength{\intextsep}{10pt} % Space between text and floats


\numberwithin{equation}{enumi}
\numberwithin{figure}{enumi}
\renewcommand{\thetable}{\theenumi}

\begin{enumerate}
\setcounter{enumi}{15}
 
%16
    \item Let $X$ have  a binomial distribution $B\brak{n,p}$ such that the sum and product of the mean and variance of $X$  are $24$ and $128$ respectively. If P$\brak{X>n-3}=\frac{k}{2^n}$
\hfill{\sbrak{July 2022}}
        \begin{multicols}{4}
            \begin{enumerate}
                \item 528
                \item 529
                \item 629
                \item 630
            \end{enumerate}
        \end{multicols}

%17
    \item A six faced die is biased such that $3\times P\brak{\text{a prime number}}=2\times P\brak{1}=6\times P\brak{\text{a composite number}}.$ Let $X$ be a random variable that counts the number of times one gets a perfect square on some throws of this die. If the die is thrown twice, then the mean of $X$ is :
\hfill{\sbrak{July 2022}}
		\begin{multicols}{1}
			\begin{enumerate}
				\item $\frac{3}{11}$
				\item $\frac{5}{11}$
				\item $\frac{7}{11}$
				\item $\frac{8}{11}$
			\end{enumerate}
		\end{multicols}

%18
    \item The angle of elevation of the top P of a vertical tower $PQ$ of height 10 from point A on the horizontal ground is 45\textdegree. Let R be a point on $AQ$ and from a point B, the angle of elevation of P is 60\textdegree. If $\angle BAQ$=30\textdegree, $AB=d$ and the area of the trapezium PQRB is $\alpha$, then the ordered pair $\brak{d,\alpha}$ is :
   \hfill{\sbrak{July 2022}}
        \begin{multicols}{1}
            \begin{enumerate}
                \item $\brak{10\brak{\sqrt{3}-1},25}$
                \item $\brak{10\brak{\sqrt{3}-1},\frac{25}{2}}$
                \item $\brak{10\brak{\sqrt{3}+1},25}$
                \item $\brak{10\brak{\sqrt{3}+1},\frac{25}{2}}$
            \end{enumerate}
        \end{multicols}

%19
    \item Let $S=\{\theta \in \brak{0,\frac{\pi}{2}} :\sum_{m=1}^{9}\sec \brak{\theta+\brak{m-1}\frac{\pi}{6}}\sec \brak{\theta+\frac{m\pi}{6}}\}$
  \hfill{\sbrak{July 2022}}
		\begin{multicols}{1}
			\begin{enumerate}
				\item $\{\frac{\pi}{6}\}$
				\item $\{\frac{2\pi}{3}\}$
				\item $\sum_{\theta \in S}\theta =\frac{\pi}{2}$
				\item $\sum_{\theta \in S}\theta =\frac{3\pi}{4}$
			\end{enumerate}
		\end{multicols}

%20
    \item If the truth value of the statement $(P\land(\neg R))\rightarrow ((\neg R) \land Q)$ is F, then the truth value of which is of the following is F ?
    \hfill{\sbrak{July 2022}}
		\begin{multicols}{1}
			\begin{enumerate}
				\item $P \lor Q \rightarrow \neg R$
				\item $R \lor Q \rightarrow \neg P$
				\item $\neg (P \lor Q) \rightarrow \neg R$
				\item $\neg (R \lor Q) \rightarrow \neg P$
			\end{enumerate}
		\end{multicols}

%21
    \item A= \myvec{4 &&-2 \\ \alpha && \beta} . If $A^2+\gamma A+ 18I=0$, then det\brak{A} is equal to :
    \hfill{\sbrak{July 2022}}
    \item The number of functions f, from the set $A= \{x \in N: x^2-10x+9 \leq 0 \}$ to the set
    $B= \{ n^2 : n\in N\}$ such that $f\brak{x}\leq
     \brak{x-3}^2+1,$ for every $x \in A, is $
    \hfill{\sbrak{July 2022}}
%23
    \item Let for the $9^{th}$ term in the binomial expansion of $\brak{3+6x}^n$, in the increasing powers of $6x$, to be the greatest for $x=\frac{3}{2}$, the least value of n is $n_0$.  If $K$ 
    is the ratio of the coefficient of $x^6$ to the coefficient of $x^3$, then $k+n_0$ is equal to:
  \hfill{\sbrak{July 2022}}

%24
    \item $\frac{2^3-1^3}{1\times7} + \frac{4^3-3^3+2^3-1^3}{2\times 11} +......+\frac{6^3-5^3+4^3-3^3+2^3-1^3}{3\times 15} +\frac{30^3-29^3+28^3-27^3+.....+2^3-1^3}{15\times 63}$ is equal to .
\hfill{\sbrak{July 2022}}
%25
    \item A water tank has the shape of a right circular cone with axis vertical and vertex downwards. Its semi-vertical angle is $\tan^{-1}\frac{3}{4}$. Water is poured in at a constant rate of 6 cubic  meter per hour. The rate, at which wet curved surface area of the tank is increasing, when the depth of the tank is 4 meters, is:
    \hfill{\sbrak{July 2022}}

    \item For the curve $C: \brak{x^2+y^2-3}+\brak{x^2-y^2-1}^5=0$, the value of $3y'-y^3y''$, at the point \brak{\alpha,\alpha
    }, $\alpha \geq 0$, on C, is equal to  :
  \hfill{\sbrak{July 2022}}
    \item Let $f\brak{x}=min\{[x-1],[x-2],......,[x-10]\}$ where $[t]$ denotes the greatest integer $\leq t$. Then $\int_{0}^{10}f\brak{x}dx$+$\int_{0}^{10}(f\brak{x})^2dx$+$\int_{0}^{10}|f\brak{x}|dx$ is equal to :
\hfill{\sbrak{July 2022}}
    \item Let f be a differentiable function satisfying $f\brak{x}=\frac{2}{\sqrt{3}}\int_{0}^{\sqrt{3}}  f\brak{\frac{\lambda^2x}{3}}d\lambda$, $x\geq0$ and $f\brak{1}=\sqrt{3}.$ If $y=f\brak{x}$ passes through the point \brak{\alpha,6}, then $\alpha$ is equal to :
   \hfill{\sbrak{July 2022}}
    \item A common tangent $T$ to the curves $C_1 : \frac{x^2}{4}+\frac{y^2}{9}=1 \text{and} C_2:\frac{x^2}{42}-\frac{y^2}{143}=1 $ does not pass through the fourth quadrant. If $T$ touches $C_1$ at \brak{x_1,y_1} and $C_2$ at \brak{x_2,y_2}, then $\abs{2x_1+x_2}$ is equal to :
 \hfill{\sbrak{July 2022}}
    \item Let, $\overrightarrow{a}$,$\overrightarrow{b}$, $\overrightarrow{c}$ be three non-coplanar vectors such that  $\overrightarrow{a}$$\times$ $\overrightarrow{b}$=4$\overrightarrow{c}$,  $\overrightarrow{b}$$\times$ $\overrightarrow{c}$=9$\overrightarrow{a}$, and  $\overrightarrow{c}$$\times$ $\overrightarrow{a}$=$\alpha$$\overrightarrow{b}$, $\alpha >0$
If $\abs{\overrightarrow{a}}+\abs{\overrightarrow{b}}+\abs{\overrightarrow{c}}=\frac{1}{36}$, then the $\alpha$ is equal to :
\hfill{\sbrak{July 2022}}

 \end{enumerate}

\end{document}
