\let\negmedspace\undefined
\let\negthickspace\undefined
\documentclass[journal]{IEEEtran}
\usepackage[a5paper, margin=8mm, onecolumn]{geometry}
%\usepackage{lmodern} % Ensure lmodern is loaded for pdflatex
\usepackage{tfrupee} % Include tfrupee package

\setlength{\headheight}{1cm} % Set the height of the header box
\setlength{\headsep}{0mm}     % Set the distance between the header box and the top of the text

\usepackage{gvv-book}
\usepackage{gvv}
\usepackage{cite}
\usepackage{amsmath,amssymb,amsfonts,amsthm}
\usepackage{algorithmic}
\usepackage{graphicx}
\usepackage{textcomp}
\usepackage{xcolor}
\usepackage{txfonts}
\usepackage{listings}
\usepackage{enumitem}
\usepackage{mathtools}
\usepackage{gensymb}
\usepackage{comment}
\usepackage[breaklinks=true]{hyperref}
\usepackage{tkz-euclide} 
\usepackage{listings}
% \usepackage{gvv}                                        
\def\inputGnumericTable{}                                 
\usepackage[latin1]{inputenc}                                
\usepackage{color}                                            
\usepackage{array}                                            
\usepackage{longtable}                                       
\usepackage{calc}                                             
\usepackage{multirow}                                         
\usepackage{hhline}                                           
\usepackage{ifthen}                                           
\usepackage{lscape}
\begin{document}

\bibliographystyle{IEEEtran}
\vspace{3cm}

\title{2023-April-10 Shift-1}
\author{AI24BTECH11005 - Prajwal Naik}
% \maketitle
% \newpage
% \bigskip
{\let\newpage\relax\maketitle}

\renewcommand{\thefigure}{\theenumi}
\renewcommand{\thetable}{\theenumi}
\setlength{\intextsep}{10pt} % Space between text and floats


\numberwithin{equation}{enumi}
\numberwithin{figure}{enumi}
\renewcommand{\thetable}{\theenumi}

\begin{enumerate}
\setcounter{enumi}{15}
 
%16
    \item An arc PQ of a circle subtends a right angle at the centre O. The midpoint of the arcPq is R.If $\overrightarrow{OP}=\overrightarrow{u}$ and $\overrightarrow{OR}=\overrightarrow{v}$ and $\overrightarrow{OQ}=\alpha \overrightarrow{u}+\beta \overrightarrow{v}$, then $\alpha, \beta^2$ are the roots of the equation :
        \begin{multicols}{1}
            \begin{enumerate}
                \item $3x^2-2x-1=0$
                \item $3x^2+2x-1=0$
                \item $x^2-x-2=0$
                \item $x^2+x-2=0$
            \end{enumerate}
        \end{multicols}

%17
    \item A square piece of tin of side 30 cm is to be made into a box without top by cutting a square from each corner
and folding up the flaps to form a box. If the volume of the box is maximum, then its surface area  is
equal to :

		\begin{multicols}{4}
			\begin{enumerate}
				\item 800
				\item  1025
				\item  900
				\item  675
			\end{enumerate}
		\end{multicols}

%18
    \item Let O be the origin and the position vector of the point P be $-\hat{i}-2\hat{j}+3\hat{k}.$ If the position vectors of the A, B
and C are $-2\hat{i}+\hat{j}-3\hat{k}$,$2\hat{i}+4\hat{j}-2\hat{k}$ and $-4\hat{i}+2\hat{j}-\hat{k}$ respectively, then the projection of the vector $\overrightarrow{OP}$ on a vector perpendicular to the vectors $\overrightarrow{AB},\overrightarrow{AC}$ is 
        \begin{multicols}{4}
            \begin{enumerate}
              \item $\frac{10}{3}$
              \item $\frac{8}{3}$
              \item $\frac{7}{3}$
              \item $3$
            \end{enumerate}
        \end{multicols}

%19
    \item If A is a $3\times 3$ matrix and $\abs{A}=2$, then $\abs{3adj\brak{\abs{3A}A^{2}}}$ is equal to :
		\begin{multicols}{1}
			\begin{enumerate}
				\item 1
    \item 2
    \item $3^{10}$
    \item None of the above
			\end{enumerate}
		\end{multicols}

%20
    \item 
		\begin{multicols}{1}
			\begin{enumerate}
				
				\item 76
    \item 74
     \item 70
      \item 72
			\end{enumerate}
		\end{multicols}

%21
    \item The negation of the statement : $(p \lor q) \land (q \lor (\neg r))$ is
    \begin{multicols}{1}
            \begin{enumerate}
              \item $((\neg p) \lor r)) \land (\neg q)$
              \item  $((\neg p) \lor (\neg q) \land(\neg r)$
              \item  $((\neg p) \lor (\neg q) \lor(\neg r)$
              \item $(p \lor r) \land  (\neg q)$
            \end{enumerate}
        \end{multicols}
%22
    \item The shortest distance between the lines $\frac{x+2}{1}=\frac{y}{-2}=\frac{z-5}{2} $ and $\frac{x-4}{1}=\frac{y-1}{2}=\frac{z+3}{0}$ is :
    \begin{multicols}{1}
            \begin{enumerate}
              \item 8
              \item  7
              \item  6
              \item 9
            \end{enumerate}
        \end{multicols}

%23
    \item If the coefficient of  $x^7$ in $\brak{ax-\frac{1}{bx^2}}^{13}$ and the coefficient of $x^{-5}$ in $\brak{ax+\frac{1}{bx^2}}^{13}$ are equal, then $a^3b^4$ is equal to :
\begin{multicols}{1}
            \begin{enumerate}
              \item 22
              \item  44
              \item  11
              \item 33
            \end{enumerate}
        \end{multicols}
%24
    \item A line segment AB of length $\lambda$ moves such that the points A and B remain on the periphery of a circle of radius $\lambda$. The locus of the point, that divides the line segment AB in the ratio 2:3, is a circle of radius :
    \begin{multicols}{1}
            \begin{enumerate}
              \item $\frac{2}{3}\lambda$
              \item   $\frac{\sqrt{19}}{7}\lambda$
              \item   $\frac{3}{5}\lambda$
              \item  $\frac{\sqrt{19}}{5}\lambda$
            \end{enumerate}
        \end{multicols}
%25
    \item For the system of linear equations \\
    $2x-y+3z=5$\\
    $3x+2y-z=7$\\
    $4x+5y+\alpha z=\beta$\\
    Which of the following is not correct?
    \begin{multicols}{1}
            \begin{enumerate}
              \item The system is inconsistent for $\alpha =-5,$ $\beta = 8$
              \item The system has infinitely many solutions for $\alpha =-6$, $\beta = 9$
              \item The system has a unique solution  for $\alpha =-5,$ $\beta = 8$
              \item The system has infinitely many solutions for $\alpha =-5,$ $\beta = 8$
            \end{enumerate}
        \end{multicols}
     
    \item Let the first term a and the common ratio r of a geometric progression be positive integers. If the sum of squares of its first three is 33033, then the sum of these terms is equal to :
    \begin{multicols}{4}
            \begin{enumerate}
              \item 210
              \item 220
              \item 231
              \item 241
            \end{enumerate}
        \end{multicols}
    
    

    \item Let P  be the point of intersection of the line $\frac{x+3}{3}=\frac{y+2}{1}=\frac{z-1}{-2}$ and the plane  $x+y+z=2$. If the distance of the point P from the plane $3x-4y+12z=32$ is q, then q and 2q are the roots of the equation :
    \begin{multicols}{4}
            \begin{enumerate}
              \item $x^2+18x-72=0$
              \item $x^2+18x+72=0$
              \item $x^2-18x-72=0$
              \item $x^2-18x+72=0$
            \end{enumerate}
        \end{multicols}
    \item Let f be a differentiable function such that $x^2f\brak{x}-x=4\int_{0}^{4} tf(t)dt.$ $f(1)=\frac{2}{3}$. Then 18f(3) is  equal to :
    \begin{multicols}{4}
            \begin{enumerate}
              \item 180
              \item 150
              \item 210
              \item 160
            \end{enumerate}
        \end{multicols}
    
    \item Let N denote the sum of the numbers obtained when two dice are rolled. If the probability that $2^{N!} < N!$ is $\frac{m}{n}$. Where $\brak{m,n}=1$ , then $4m-3n$ equal to:
    \begin{multicols}{4}
            \begin{enumerate}
              \item 180
              \item 150
              \item 210
              \item 160
            \end{enumerate}
        \end{multicols}
        \item If $I(x)=\int_ e^{\sin{x}^2} (\cos{x}\sin{2x}-\sin{x}) dx$ and $I\brak{0}=1$, then $I(\frac{\pi}{3})$ is equal to :
        \begin{multicols}{4}
            \begin{enumerate}
              \item $e^{\frac{3}{4}}$
              \item $-e^{\frac{3}{4}}$
            \item $\frac{1}{2}e^{\frac{3}{4}}$
              \item $-\frac{1}{2}e^{\frac{3}{4}}$
            \end{enumerate}
        \end{multicols}
 \end{enumerate}

\end{document}