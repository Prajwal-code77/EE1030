\let\negmedspace\undefined
\let\negthickspace\undefined
\documentclass[journal]{IEEEtran}
\usepackage[a5paper, margin=10mm, onecolumn]{geometry}
%\usepackage{lmodern} % Ensure lmodern is loaded for pdflatex
\usepackage{tfrupee} % Include tfrupee package

\setlength{\headheight}{1cm} % Set the height of the header box
\setlength{\headsep}{0mm}     % Set the distance between the header box and the top of the text

\usepackage{gvv-book}
\usepackage{gvv}
\usepackage{cite}
\usepackage{amsmath,amssymb,amsfonts,amsthm}
\usepackage{algorithmic}
\usepackage{graphicx}
\usepackage{textcomp}
\usepackage{xcolor}
\usepackage{txfonts}
\usepackage{listings}
\usepackage{enumitem}
\usepackage{mathtools}
\usepackage{gensymb}
\usepackage{comment}
\usepackage[breaklinks=true]{hyperref}
\usepackage{tkz-euclide} 
\usepackage{listings}
% \usepackage{gvv}                                        
\def\inputGnumericTable{}                                 
\usepackage[latin1]{inputenc}                                
\usepackage{color}                                            
\usepackage{array}                                            
\usepackage{longtable}                                       
\usepackage{calc}                                             
\usepackage{multirow}                                         
\usepackage{hhline}                                           
\usepackage{ifthen}                                           
\usepackage{lscape}
\begin{document}

\bibliographystyle{IEEEtran}
\vspace{3cm}

\title{Electrical Engineering-2011}
\author{AI24BTECH11005 - Bhukya Prajwal Naik
}
% \maketitle
% \newpage
% \bigskip
{\let\newpage\relax\maketitle}

\renewcommand{\thefigure}{\theenumi}
\renewcommand{\thetable}{\theenumi}
\setlength{\intextsep}{10pt} % Space between text and floats


\numberwithin{equation}{enumi}
\numberwithin{figure}{enumi}
\renewcommand{\thetable}{\theenumi}


\begin{enumerate}
    \setcounter{enumi}{39}
 

    \item A capacitor is made with a polymetric dielectric having an $\epsilon_r$ of 2,26 and a dielectric breakdown strength of 50$\frac{kV}{cm}$ .The permitivity of free space is 8.85 $\frac{pF}{m}$ . If the rectangular plates of the capacitor have a width of 20cm and a length of 40cm , then the maximum electric charge in the capacitor is
    \begin{multicols}{4}
			\begin{enumerate}
\item 2$\mu C$
\item 4$\mu C$
\item 8$\mu C$
\item 10$\mu C$
        \end{enumerate}
		\end{multicols}


    \item The response of a linear time invariant system to an implulse $\delta (t)$. under initially relaxed condition is $h(t)=e^-1 +e^-2t$. The response of this system for a unit step input $u(t)$ is


		\begin{multicols}{4}
			\begin{enumerate}
	\item  $u(t) +h(t)$
\item $u(t)*h(t)$
\item$(1.5-h(t))u(t)$
\item $h(t)((\delta  (t) + u(t))$
			\end{enumerate}
		\end{multicols}


    \item The direct axis and quadrature axis reactances of a salient pole alternator are 1.2p.u and 1.0p.u respectively . The armature resistance is negligible .If this alternator is delivering rated kVA at upf and at rated voltage then its power angle is 
        
            \begin{enumerate}
             \item $30\degree$
\item $45\degree$.
\item  $60\degree$
\item $90\degree$            \end{enumerate}



      


    \item A $\frac{9}{2}$ digit DMM has the error specification as 0.2\% of reading 10 counts .If a dc voltage of 100V is read on its 200V full scale , the maximum error that can be expected  in the reading is 
		\begin{multicols}{4}
			\begin{enumerate}
	\item $0.1 \%$
\item $0.2 \%$
\item $0.3 \%$
\item $0.4 \%$

	\end{enumerate}
		\end{multicols}{1}

 
    \item A three-bus network is shown in the figure below indicating the p.u impedance of each element
    
     \begin{figure}[!ht]
\centering
\resizebox{0.8\textwidth}{!}{%
\begin{circuitikz}
\tikzstyle{every node}=[font=\small]

\draw (1.75,10) to (3.75,10) node[ground]{};
\draw (1.75,11.5) to[L ] (3.75,11.5);
\draw (1.75,11.5) to[short] (1.75,10);
\draw (3.5,11.75) to[short] (3.5,11.25);
\draw (3.75,11.5) to[L ] (5.75,11.5);
\draw (5.75,11.5) to[C] (7.75,11.5);
\draw (7.75,11.5) to[L ] (9.75,11.5);
\draw (9.75,11.5) to (11.75,11.5) node[ground]{};
\node [font=\small] at (2.5,12.5) {j 0.1};
\draw  (3.5,12.5) circle (0.25cm);
\node [font=\small] at (3.5,12.5) {1};
\draw [short] (5.75,12) -- (5.75,11.25);
\draw  (5.75,12.25) circle (0.25cm);
\node [font=\small] at (5.75,12.25) {2};
\node [font=\small] at (6.75,10.75) {-j 0.08};
\draw [short] (7.75,11.75) -- (7.75,11.25);
\draw  (7.75,12) circle (0.25cm);
\node [font=\small] at (7.75,12) {3};
\node [font=\small] at (9,11) {j 0.1};
\end{circuitikz}
}%

\label{fig:my_label}
\end{figure}

    \newpage
		The bus admittance matrix Y-bus of the network is 
			\begin{enumerate}
				
	\item j$\myvec{0.3 & -0.2 & 0 \\ -0.2 & 0.12 & 0.08 \\ 0 & 0.08 & 0.02}$
\item j$\myvec{-15 & 5 & 0 \\ 5 & 7.5 & -12.5\\ 0& -12.5 & 2.5}$
\item j$\myvec{0.1 & 0.2 & 0 \\ 0.2 & 0.12 & -0.08\\ 0& -0.08 & 0.10}$
\item j$\myvec{10& 5 & 0 \\ 5 & 7.5 & 12.5\\ 0& 12.5 & -10}$
			\end{enumerate}
		


    \item A two-loop position control system is shown below
    \begin{figure}[!ht]
\centering
\resizebox{1\textwidth}{!}{%
\begin{circuitikz}
\tikzstyle{every node}=[font=\normalsize]
\draw  (5.75,13.75) circle (0.75cm);
\draw [->, >=Stealth] (3.75,13.75) -- (5,13.75);
\node [font=\small] at (3.25,13.75) {R(s)};
\draw (5.25,13.25) to[short] (6.25,14.25);
\node [font=\small] at (5.25,13.75) {+};
\node [font=\small] at (5.75,13.5) {-};
\draw [->, >=Stealth] (6.5,13.75) -- (7.75,13.75);
\draw  (8.5,14) circle (0.75cm);
\draw (8,14.5) to[short] (9,13.5);
\draw (8,13.5) to[short] (9,14.5);
\node [font=\large] at (8,14) {+};
\node [font=\large] at (8.5,13.75) {-};
\draw  (11,14.25) rectangle (13.25,13);
\draw [->, >=Stealth] (9.25,14.25) -- (11,13.5);
\node [font=\large] at (11.75,14.75) {Motor};
\node [font=\large] at (12,13.75) {1/s(s+1)};
\draw (13.25,13.75) to[short] (17.75,13.75);
\draw [->, >=Stealth] (17.75,13.75) -- (18,13.75);
\draw (17,13.75) to[short] (17,9.5);
\draw (17,9.5) to[short] (5.75,9.5);
\draw (5.75,13) to[short] (5.75,9.5);
\draw [->, >=Stealth] (5.75,12.25) -- (5.75,13);
\draw [->, >=Stealth] (8.25,12) -- (8.25,13.25);
\draw (8.25,12) to[short] (10.25,12);
\draw  (10.75,12) rectangle (12.25,11);
\draw [short] (10.25,12) -- (10.75,12);
\node [font=\small] at (11.25,11.5) {K s};
\node [font=\normalsize] at (11.5,10.25) {Tacho-generator};
\draw (12.25,12) to[short] (15,12);
\draw (15,13.75) to[short] (15,12);
\draw [->, >=Stealth] (13.25,12) -- (13,12);
\draw [->, >=Stealth] (17,12.25) -- (17,11.75);
\draw [->, >=Stealth] (13,9.5) -- (12.25,9.5);
\draw (5.25,14.5) to[short] (5.75,14);
\draw (5.25,14.5) to[short] (6.25,13.5);
\draw (6,13.75) to[short] (6.5,13.25);
\node [font=\normalsize] at (18.5,13.75) {Y(s)};
\end{circuitikz}
}%

\label{fig:my_label}
\end{figure}

    The  gain k of the Tacho generator influences mainly the 
    
            \begin{enumerate}
        \item peak overshoot
        \item natural frequency of oscillation
        \item phase shift of the closed loop transfer function at very low frequencies $(\omega \rightarrow 0)$
        \item phase shift of the closed loop transfer function at very low frequencies $(\omega \rightarrow \infty)$
        \end{enumerate}
      

    \item A two-bit counter circuit is shown below
    \begin{figure}[!ht]
\centering
\resizebox{1\textwidth}{!}{%
\begin{circuitikz}
% Set font size globally for all nodes
\tikzstyle{every node}=[font=\normalsize]

% Draw components and connections
\draw (8.5,13.25) rectangle (13,10.25);
\draw (17.75,12.75) rectangle (21.5,10.25);
\draw (8.5,12.5) -- (6.5,12.5) -- (6.5,14.5) -- (22,14.5) -- (22.5,14.5);
\draw (6.75,11.5) -- (8.5,11.5) -- (6.75,9);
\draw (6,9) -- (16,9) -- (16,10.75);
\draw (8.5,10.75) -- (7.5,10.75) -- (7.5,9.75) -- (22,9.75) -- (22,13.25);
\draw (21.5,12.25) -- (22,12.25);
\draw (22.5,14.5) -- (24,14.5) -- (24,11) -- (21.25,11);
\draw (16,10.75) -- (16,11.5) -- (17.75,11.5);
\draw [->, >=Stealth] (16,11.5) -- (18,11.5);
\draw (13,12.5) -- (18,12.5);
\draw [->, >=Stealth] (14.25,13.5) -- (14.25,14);
\draw (14.25,13.5) -- (14.25,12.5);
\draw [->, >=Stealth] (22,13) -- (22,13.75);
\draw [->, >=Stealth] (8.5,11.5) -- (8.75,11.5);

% Add labels
\node at (5.75,9.25) {CLK};
\draw (6,9) -- (6.75,9);
\node at (9,12.75) {J};
\node at (9,10.75) {K};
\node at (12.75,12.75) {Q};
\draw (12.5,11) -- (12.75,11);
\node at (12.5,10.75) {Q};
\node at (14.5,13.5) {Q\_A};
\node at (18.25,12.5) {T};
\node at (21.25,12.5) {Q};
\draw (21,11.25) -- (20.5,11.25);
\node at (20.75,11) {Q};
\node at (22.25,13.5) {Q\_B};
\end{circuitikz}
}%
\label{fig:my_label}
\end{figure}

    In the state $Q_A Q_B$ of the counter at the clock time $t_n$ is 10, the the state $Q_A Q_B$ of the counter at $t_n +3$(after three clock cycles)  will be 
    
            \begin{enumerate}
             \item 00
\item 01
\item 10
\item 11 
              \end{enumerate}
        

    \item A clipper circuit is shown below.
    \begin{figure}[!ht]
\centering
\resizebox{1\textwidth}{!}{%
\begin{circuitikz}
\tikzstyle{every node}=[font=\normalsize]

\draw (11,11) to[sinusoidal voltage source, sources/symbol/rotate=auto] (13,11);
\draw (11,11) to[R] (11,14);
\draw (13.75,14.25) to[empty Zener diode] (11,14.25);
\draw (11,14) to[short] (11,13.25);
\draw (11,13.25) to[short] (11,14.25);
\draw (13.75,14.25) to[short] (13.75,11);
\draw (13,11) to[short] (13.75,11);
\draw (11,14.25) to[short] (11,16);
\draw (13,16) to[short, -o] (11,16) ;
\draw (13.5,16) to[D] (13,16);
\draw (13.5,16) to[short] (14.5,16);
\draw (16,16) to[battery1] (16.75,16);
\draw (14.5,16) to[short] (16,16);
\draw (16.75,16) to[short] (16.75,14.25);
\draw (13.75,14.25) to[short] (16.75,14.25);
\draw (11,16) to[short, -o] (11,17.75) ;
\draw (16.75,16) to[short, -o] (16.75,17.75) ;
\draw [->, >=Stealth] (13,18) -- (11.25,18);
\draw [->, >=Stealth] (13.75,18) -- (16.5,18);
\draw [->, >=Stealth] (11.75,10.25) -- (11,10.25);
\draw [->, >=Stealth] (12.5,10.25) -- (14,10.25);
\node [font=\normalsize] at (10.25,12.25) {1k};
\node [font=\normalsize] at (12,10.25) {$V_1$};
\node [font=\normalsize] at (16.25,16.75) {5V};
\node [font=\normalsize] at (12.25,13.75) {$V_z=10V$};
\node [font=\normalsize] at (13.25,18) {$V_0$};
\node [font=\normalsize] at (13.25,16.75) {D};
\end{circuitikz}
}%

\label{fig:my_label}
\end{figure}

    Assuming forward voltage drops of the diodes to be 0.7 V, the input-output transfer characteristics of the circuit is

 \begin{enumerate}
 \newpage
\item \begin{figure}[!ht]
\centering
\resizebox{1\textwidth}{!}{%
\begin{circuitikz}
\tikzstyle{every node}=[font=\normalsize]

\draw [short] (9.25,16.25) -- (9.25,10.5);
\draw [short] (7,12) -- (18,12);
\draw [short] (9.25,12) -- (11,15);
\draw [short] (11,15) -- (15,15);
\draw [dashed] (9.25,15) -- (11,15);
\node [font=\normalsize] at (8.25,15.5) {$V_0$};
\draw [->, >=Stealth] (8.25,16) -- (8.25,17);
\draw [->, >=Stealth] (9.25,16.25) -- (9.25,17.5);
\draw [->, >=Stealth] (18,12) -- (19.25,12);
\draw [->, >=Stealth] (13.25,11.25) -- (15.25,11.25);
\node [font=\normalsize] at (15.75,11.25) {$V_1$};
\draw [dashed] (11,15) -- (11,12);
\node [font=\normalsize] at (9,15) {4.3};
\node [font=\normalsize] at (11,11.5) {4.3};
\end{circuitikz}
}%

\label{fig:my_label}
\end{figure}
\newpage
\item \begin{figure}[!ht]
\centering
\resizebox{1\textwidth}{!}{%
\begin{circuitikz}
\tikzstyle{every node}=[font=\normalsize]

\draw [short] (9.25,16.25) -- (9.25,10.5);
\draw [short] (7,12) -- (18,12);
\node [font=\normalsize] at (8.25,15.5) {$V_0$};
\draw [->, >=Stealth] (8.25,16) -- (8.25,17);
\draw [->, >=Stealth] (9.25,16.25) -- (9.25,17.5);
\draw [->, >=Stealth] (18,12) -- (19.25,12);
\draw [->, >=Stealth] (13.25,11.25) -- (15.25,11.25);
\node [font=\normalsize] at (15.75,11.25) {$V_1$};
\node [font=\normalsize] at (11,11.5) {4.3};
\draw [dashed] (9.25,16.75) -- (12.75,16.75);
\draw [short] (12.75,16.75) -- (16,16.75);
\draw [short] (12.75,16.75) -- (11,14.75);
\draw [dashed] (11,14.75) -- (11,12.25);
\draw [short] (11,14.75) -- (7.25,14.75);
\draw [dashed] (12.75,16.75) -- (12.75,12);
\node [font=\normalsize] at (8.75,14.5) {4..3};
\node [font=\normalsize] at (12.75,11.75) {10};
\end{circuitikz}
}%

\label{fig:my_label}
\end{figure}

\newpage
\item \begin{figure}[!ht]
\centering
\resizebox{1\textwidth}{!}{%
\begin{circuitikz}
\tikzstyle{every node}=[font=\normalsize]

\draw [short] (9.25,16.25) -- (9.25,10.5);
\draw [short] (7,12) -- (18,12);
\node [font=\normalsize] at (8.25,15.5) {$V_0$};
\draw [->, >=Stealth] (8.25,16) -- (8.25,17);
\draw [->, >=Stealth] (9.25,16.25) -- (9.25,17.5);
\draw [->, >=Stealth] (18,12) -- (19.25,12);
\draw [->, >=Stealth] (13.25,11.25) -- (15.25,11.25);
\draw [dashed] (9.25,16.75) -- (12.75,16.75);
\draw [short] (12.75,16.75) -- (17,16.75);
\draw [short] (12.75,16.75) -- (9.25,12);
\draw [short] (9.25,12) -- (9,11.75);
\draw [short] (9,11.75) -- (6,11.75);
\node [font=\normalsize] at (15.75,11.25) {$V_1$};
\node [font=\normalsize] at (12.75,11.75) {5.7};
\node [font=\normalsize] at (9.5,11.75) {-0.7};
\node [font=\normalsize] at (9,12.25) {-0.7};
\node [font=\normalsize] at (8.75,16.75) {5.7};
\draw [dashed] (12.75,16.75) -- (12.75,12.25);
\end{circuitikz}
}%

\label{fig:my_label}
\end{figure}


\item \begin{figure}[!ht]
\centering
\resizebox{1\textwidth}{!}{%
\begin{circuitikz}
\tikzstyle{every node}=[font=\normalsize]

\draw [short] (9.25,16.25) -- (9.25,10.5);
\draw [short] (7,12) -- (18,12);
\node [font=\normalsize] at (8.25,15.5) {$V_0$};
\draw [->, >=Stealth] (8.25,16) -- (8.25,17);
\draw [->, >=Stealth] (9.25,16.25) -- (9.25,17.5);
\draw [->, >=Stealth] (18,12) -- (19.25,12);
\draw [->, >=Stealth] (13.25,11.25) -- (15.25,11.25);
\draw [dashed] (9.25,16.75) -- (12.75,16.75);
\draw [short] (12.75,16.75) -- (17,16.75);
\node [font=\normalsize] at (15.75,11.25) {$V_1$};
\draw [short] (7.75,10.5) -- (6.25,10.5);
\draw [dashed] (12.75,16.75) -- (12.75,12);
\draw [dashed] (7.75,10.5) -- (7.75,12);
\draw [dashed] (7.75,10.5) -- (9.25,10.5);
\draw (9.25,10.5) to[short] (9.25,8.75);
\node [font=\normalsize] at (9,16.75) {10};
\node [font=\normalsize] at (12.75,11.75) {10};
\node [font=\normalsize] at (7.75,12.25) {-5.7};
\node [font=\normalsize] at (9.5,10.5) {-5.7};
\draw (6.25,10.5) to[short] (3.5,10.5);
\draw (7.75,10.5) to[short] (9.25,12);
\draw (9.25,12) to[short] (9.25,15.75);
\draw [short] (12.75,16.75) -- (9.25,12);
\end{circuitikz}
}%

\label{fig:my_label}
\end{figure}

            \end{enumerate}
        

\textbf{Common Data for the below two questions }
The input voltage given to a converter is $v_i=100\sqrt{2} \sin(100\pi t)V$, The current drawn by the converter is $i_1=(10\sqrt{2} \sin (100\pi t-\frac{\pi}{3})+5\sqrt{2} \sin (300\pi t+\frac{\pi}{4})+2\sqrt{2}\sin (500\pi t-\frac{\pi}{6}))A$
    \item Then the power factor  of the converter is 
    \begin{multicols}{4}
            \begin{enumerate}
        \item 0.31
\item 0.44
\item  0.5
\item 0.71
            \end{enumerate}
        \end{multicols}
 

 
    \item Then the active power drawn by the converter is
    \begin{multicols}{4}
            \begin{enumerate}
              \item 181 W
              \item  500 W
              \item  707 W
              \item 887 W
            \end{enumerate}
        \end{multicols}
        \textbf{Common Data for Questions 50 and 51:}
        An RLC circuit with relevant data is given below:
        \begin{figure}[!ht]
\centering
\resizebox{1\textwidth}{!}{%
\begin{circuitikz}
\tikzstyle{every node}=[font=\normalsize]

\draw (10.5,10.75) to[sinusoidal voltage source, sources/symbol/rotate=auto] (12.5,10.75);
\draw [short] (10.5,10.75) -- (8.25,10.75);
\draw [short] (8.25,10.75) -- (8.25,12.75);
\draw [short] (8.25,12.75) -- (8.25,15.5);
\draw (8.25,15.5) to[C] (12.75,15.5);
\draw (12.75,15.5) to[short] (15.25,15.5);
\draw (15.25,15.5) to[short] (15.25,11);
\draw (12.5,10.75) to[short] (15.25,10.75);
\draw (15.25,11.25) to[short] (15.25,10.75);
\draw (8.25,13) to[R] (11.75,13);
\draw (11.75,13) to[L ] (15.25,13);
\draw [->, >=Stealth] (8.5,13) -- (8.75,13);
\draw [->, >=Stealth] (8.25,15.5) -- (9,15.5);
\draw [->, >=Stealth] (8.25,11.75) -- (8.25,12);
\node [font=\normalsize] at (8,11.75) {$I_s$};
\node [font=\normalsize] at (8.5,13.5) {$I_{RL}$};
\node [font=\normalsize] at (10,13.5) {R};
\node [font=\normalsize] at (13.25,13.75) {L};
\node [font=\normalsize] at (10.5,16.25) {C};
\node [font=\normalsize] at (8.75,16) {$I_C$};
\end{circuitikz}
}%

\label{fig:my_label}
\end{figure}

        $V_S=1\langle 0 V$
        $I_S=\sqrt{2}\langle \frac{\pi}{4}A$
        $I_{RL}=\sqrt{2}\langle \frac{-\pi}{4}A$
    
     
    \item The power dissipated in the resistor R
    \begin{multicols}{4}
            \begin{enumerate}
              \item  0.5 W
              \item 1 W
              \item $\sqrt{2} W$
              \item 2W 
            \end{enumerate}
        \end{multicols}
    
    

    \item  The current $I_C$ in the figure above is
   \begin{multicols}{4}
            \begin{enumerate}
            \item -j 2A
            \item $-j \frac{1}{\sqrt{2}A}$
            \item  $+j \frac{1}{\sqrt{2}A}$
            \item +j 2A
            \end{enumerate}
             \end{multicols}
       
    
        
 
 \end{enumerate}

\end{document}
